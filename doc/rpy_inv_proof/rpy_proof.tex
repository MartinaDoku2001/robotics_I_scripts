\documentclass{article}

\author{Dario Loi}

\date{\today}
\title{RPY Singularity Inverse Proof}

\usepackage{amsmath}
\usepackage{amssymb}
\usepackage{amsthm}
\usepackage{mathtools}

% define definitions, theorems, etc.
\theoremstyle{definition}
\newtheorem{definition}{Definition}[section]
\newtheorem{theorem}{Theorem}[section]

\DeclareMathOperator{\atan2}{atan2}

\begin{document}
\maketitle

\section{Proof}
\begin{definition}[RPY Inverse]
    A singularity occurs in an inverse problem of a fixed-axis RPY matrix if the pitch angle
    $\theta$ is equal to $\pm \frac{\pi}{2}$. We can still extract the roll and yaw angles as a sum/difference,
    but we cannot recover the individual angles precisely.

    The formula for these combinations is:

    \begin{equation}
        \begin{cases}
            \phi - \psi = \atan2 \left\{ \texttt{R[0][2]}, \texttt{R[1][2]} \right\} \quad  & \text{if } \theta = \frac{\pi}{2}  \\
            \phi + \psi = \atan2 \left\{ \texttt{-R[1][2]}, \texttt{R[1][1]} \right\} \quad & \text{if } \theta = -\frac{\pi}{2}
        \end{cases}
    \end{equation}

    Where \texttt{R} is the rotation matrix, indexed starting from 0.\ (i.e.:\ \texttt{R[0][2]} is the first row, third column).
\end{definition}

\begin{proof}[RPY Inverse Proof]
    \begin{align}
        R(\psi, \theta, \phi)         & = \begin{bmatrix}
                                              c\phi c\theta & c\phi s\theta s\psi - s\phi c\psi & c\phi s\theta c\psi + s\phi s\psi \\
                                              s\phi c\theta & s\phi s\theta s\psi + c\phi c\psi & s\phi s\theta c\psi - c\phi s\psi \\
                                              -s\theta      & c\theta s\psi                     & c\theta c\psi
                                          \end{bmatrix} \\
        \intertext{Considering the case where $\theta = \frac{\pi}{2}$:}
        R(\psi, \frac{\pi}{2}, \phi)  & = \begin{bmatrix}
                                              0  & c\phi s\psi - s\phi c\psi & c\phi c\psi + s\phi s\psi \\
                                              0  & s\phi s\psi + c\phi c\psi & s\psi c\phi - c\psi s\phi \\
                                              -1 & 0                         & 0
                                          \end{bmatrix}                            \\
        \intertext{Now consider two trigonometric properties:}
        \sin(\phi - \psi)             & = \underbrace{\sin(\phi)\cos(\psi) - \cos(\phi)\sin(\psi)}_{\texttt{R[0][2]}}           \\
        \cos(\phi - \psi)             & = \underbrace{\cos(\phi)\cos(\psi) + \sin(\phi)\sin(\psi)}_{\texttt{R[1][2]}}           \\
        \intertext{We can now solve for $\phi - \psi$ using the inverse tangent function:}
        \phi - \psi                   & = \atan2 \left\{ \texttt{R[0][2]}, \texttt{R[1][2]} \right\}
        \intertext{Similarly, for $\theta = -\frac{\pi}{2}$:}
        R(\psi, -\frac{\pi}{2}, \phi) & = \begin{bmatrix}
                                              0 & -c\phi s \psi - s \phi c\psi & s \phi s \psi - c \phi c \psi    \\
                                              0 & c\phi c \psi - s \phi s\psi  & -(s \phi c \psi + c \phi s \psi) \\
                                              1 & 0                            & 0
                                          \end{bmatrix}                   \\
        \intertext{Through the dual of the previous properties:}
        \sin(\phi + \psi)             & = \underbrace{\sin(\phi)\cos(\psi) + \cos(\phi)\sin(\psi)}_{\texttt{-R[1][2]}}          \\
        \cos(\phi + \psi)             & = \underbrace{\cos(\phi)\cos(\psi) - \sin(\phi)\sin(\psi)}_{\texttt{R[1][1]}}           \\
        \intertext{We can now solve for $\phi + \psi$ using the inverse tangent function:}
        \phi + \psi                   & = \atan2 \left\{ \texttt{-R[1][2]}, \texttt{R[1][1]} \right\}
    \end{align}
    Additionally, one can freely choose an assignment for either $\phi$ or $\psi$, and then solve for the other angle.
\end{proof}
\end{document}
