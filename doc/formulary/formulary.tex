
%author: Martina Doku
%date: 2019-10-15

\documentclass[11pt]{article}
\usepackage{amsmath}
\usepackage{hyperref}
\DeclareMathOperator{\atan2}{atan2}

\begin{document}
\title{Robotics 1 \\ Exercise Solver}
\maketitle
\tableofcontents
\section{DC motors}\label{sec:dc_motors}
\subsection{Electrical and mechanical balance}
\begin{align}
&V_a(t) = R_ai_a(t) + L_a\frac{di_a(t)}{dt} + v_{emf}(t)\\
&v_{emf}(t) = k_v\omega(t)
\end{align}
in control domain:
\begin{align}
 &V_a= (R_a + sL_a)I_a+ V_{emf}\\
    &V_{emf} = k_v\omega
\end{align}
where $V_a$ is the voltage applied to the motor, $R_a$ is the armature 
resistance, $L_a$ is the armature inductance, $i_a$ is the armature 
current, $v_{emf}$ is the back emf, $k_v$ is the back emf constant
 and $\omega$ is the angular velocity of the motor.
\begin{align}
    &\tau_m(t) = I_m(t)\frac{d\omega(t)}{dt} + F_m\omega(t)+ \tau_{load}(t)\\
    &\tau_m(t) = k_t i_a(t)
\end{align}
in control domain:
\begin{align}
    &T_m= (sI_m + F_m)\omega + T_{load}\\
    &T_m = k_t I_a
\end{align}
where $\tau_m$ is the motor torque, $I_m$ is the motor inertia,
 $F_m$ is the motor friction, $\tau_{load}$ is the load torque
  and $k_t$ is the torque constant.
\textbf{Note: $k_v = k_t$ numerically!}
\subsection{Reduction ratio}
The reduction ratio of a the ransmission chain is the product of the
 reduction ratios of the single elements of the chain:
\begin{equation}
    \eta = \sum_{i=1}^n \eta_i
\end{equation}
\subsubsection{Harmonic drives}
\begin{align}
    &\eta  = \frac{\#theet_{FS}}{\#theet_{CS}-\#theet_{FS}} = \frac{\#theet_{FS}}{2}\\
    &\#theet_{FS} = \#theet_{CS} - 2
\end{align}
\subsubsection{Standard gears}
Given two gears of radius $r_1$ and $r_2$ the reduction ratio is:
\begin{equation}
    \eta = \frac{r_2}{r_1}
\end{equation}
\subsection{Optimal reduction ratio}
\begin{equation}
    \eta_{opt} = \sqrt{\frac{J_{load}}{J_{motor}}}
\end{equation}
\subsection{Optimal torque}
We impose the relation between the angular acceleration 
of the load and the motor:
\begin{equation}
    \dot{\dot{\theta_{m}}} = \eta\dot{\dot{\theta_{l}}}
\end{equation}
\begin{equation}
    \tau_{m} = J_{m}*\dot{\dot{\theta_m}} +\frac{1}{\eta}(J_{l}*\dot{\dot{\theta_l}})
\end{equation}
\section{Encoders}\label{sec:encoders}
\subsection{Absolute encoders}
The resolution of an absolute encoder is given by:
\begin{equation}
    res = \frac{2\pi}{2^{N_t}}
\end{equation}
where $N_t$ is the number of bits of the encoder.
\textbf{Note: the resolution changes from base to link end!}
\begin{equation}
    res_{base} = res_{link}/L
\end{equation}
where L is the length of the link.
\subsection{Incremental encoders}
The resolution of an incremental encoder is given by:
\begin{equation}
    rse = \frac{2\pi}{2^{N_t}}
\end{equation}
The number of bit of the encoder is given by:
\begin{equation}
    N_t = \log_2(N_p)
\end{equation}
where $N_p$ is the number of pulses per turn of the encoder.
\subsection{Multi-turn encoders}
The number of bits to count the turns in a multi-turn encoder is given by:
\begin{equation}
    N_t = \log_2(N_{turns})
\end{equation}
where $N_{turns}$ is the number of turns of the encoder.
The number of turns of the encoder is given by:
\begin{equation}
    N_{turns} = \frac{\Delta\theta_{max}*n_r}{2\pi}
\end{equation}
where $\delta\theta_{max}$ is the maximum angle of the encoder and $n_r$ is the reduction ratio.
\section{Rotation Matrices}\label{sec:rotation}
\subsection{Check if R is a rotation matrix}\label{sec:check_rotation}
To check if R is a rotation matrix we have to check:
\begin{itemize}
\item det(R) = 1
\item Orthogonality: $R^TR = I$
\item Normality: for each column $R_i$ of R, $||R_i|| = 1$
\end{itemize}
\subsection{General Rotation}
\begin{equation}
^{A}R_B = \begin{bmatrix}
x_A  x_B & y_A x_B & z_A x_B \\
x_A  y_B & y_A y_B & z_A y_B \\
x_A  z_B & y_A z_B & z_A z_B
\end{bmatrix}
\end{equation}

\subsection{Rotation direct problem}
To find R from $\theta$ and \textbf{r} we use the Rodrigues' rotation formula:
\begin{equation}
R(\theta, r)= rr^T+ (I-rr^T)\cos(\theta) + (S(r))\sin(\theta)
\end{equation}
where $S(r)$ is the skew-symmetric matrix of \textbf{r}:
\begin{equation}
S(r) = \begin{bmatrix}
0 & -r_3 & r_2 \\
r_3 & 0 & -r_1 \\
-r_2 & r_1 & 0
\end{bmatrix}
\end{equation}
\subsection{Rotation inverse problem}
To find $\theta$ and \textbf{r} from R we fisrt check if there is
a singularity:
\begin{equation}
\sin(\theta) = \frac{1}{2}\left(\sqrt{(R_{32}-R_{23})^2+(R_{13}-R_{31})^2+(R_{21}-R_{12})^2}\right)
\end{equation}
\subsubsection{singularity (hence $\sin(\theta) = 0$)}
If it is a singularity we can find \textbf{r} and $\theta$:
if $\theta$ is 0: there is no solution for r.
if $\theta$ is $\pm\pi$:\\
we set $sin(\theta) = 0$, $\cos(\theta) = -1$ and we find \textbf{r}:
\begin{equation}
\textbf{r} = \begin{bmatrix}
\pm \sqrt{\frac{R_{11}+1}{2}} \\
\pm \sqrt{\frac{R_{22}+1}{2}} \\
\pm \sqrt{\frac{R_{33}+1}{2}}
\end{bmatrix}
\end{equation}
To decide the signs of the elements of \textbf{r} we can use the following criteria:
\begin{itemize}
\item $r_x r_y = R_{12}/2$
\item $r_x r_z = R_{13}/2$
\item $r_y r_z = R_{23}/2$
\end{itemize}
\subsubsection{not singularity}
If the singularity is not present we can find theta and \textbf{r}:\\
\textbf{Note: we obtain two solutions for $\theta$ and cosequently r}
\begin{align}
\cos(\theta) &= \left(R_{11}+R_{22}+R_{33}-1\right)\\
\sin(\theta) &= \pm\sqrt{(R_{32}-R_{23})^2+(R_{13}-R_{31})^2+(R_{21}-R_{12})^2}\\
\theta &= \atan2\left(\sin\theta,\cos\theta\right) \in (-\pi,\pi]
\end{align}
\begin{equation}
\textbf{r} = \frac{1}{2\sin(\theta)}\begin{bmatrix}
R_{32}-R_{23} \\
R_{13}-R_{31} \\
R_{21}-R_{12}
\end{bmatrix}
\end{equation}
\section{Euler}
\subsection{Euler direct problem}
To find R from $\phi$, $\theta$ and $\psi$ around axis X,Y,Z we use the 
following formula:
\begin{equation}
R(\psi,\theta,\phi) = R_x(\phi)R_y(\theta)R_z(\psi)
\end{equation}
\subsection{Inverse Problem}
Given a rotation matrix R we can find $\phi$, $\theta$ and $\psi$:
Fisrt check if there is a singularity (if $\theta = 0$ or $\pm \pi$).
\subsubsection{singularity (hence $R_{13}^2+R_{23}^2=0$)}
If it is a singularity we can find $\phi+\psi$ and $\phi-\psi$
\subsection{not singularity}
If it is not a singularity we can find $\phi$, $\theta$ and $\psi$:
\begin{align}
\theta &= \atan2\left(\pm\sqrt{R_{13}^2+R_{23}^2},R_{33}\right)\\
\phi &= \atan2\left(R_{13}/\sin(\theta),-R_{23}/\sin(\theta)\right)\\
\psi &= \atan2\left(R_{31}/\sin(\theta),R_{32}/\sin(\theta)\right)
\end{align}
\section{Roll Pitch Yawn}\label{sec:rpy}
\subsection{RPY direct problem}
To find R from $\psi$, $\theta$ and $\phi$ we use the following formula:
\begin{equation}
R(\phi,\theta,\psi) = R_z(\phi)R_y(\theta)R_x(\psi)
\end{equation}
\textbf{Note: the order of the angle is reversed!}
\subsection{Inverse Problem}
Given a rotation matrix R we can find $\psi$, $\theta$ and $\phi$:
Fisrt check if there is a singularity (if $R_{32}^2+R_{33}^2 = 0$).
\subsubsection{singularity (hence $R_{32}^2+R_{33}^2 = 0$)}
If it is a singularity we can find $\phi+\psi$ and $\phi-\psi$:
\begin{align}
\phi+\psi &= \atan2\left(-R_{23},R_{13}\right) or \\
\phi+\psi &= \atan2\left(-R_{12},R_{22}\right)\\
\phi-\psi &= \atan2\left(R_{31},\pm\sqrt{R_{32}^2+R_{33}^2}\right)
\end{align}
\subsubsection{not singularity}
If it is not a singularity we can find $\psi$, $\theta$ and $\phi$:
\begin{align}
\theta &= \atan2\left(-R_{31},\pm\sqrt{R_{32}^2+R_{33}^2}\right)\\
\phi &= \atan2\left(R_{21}/\cos(\theta),R_{11}/\cos(\theta)\right)\\
\psi &= \atan2\left(R_{32}/\cos(\theta),R_{33}/\cos(\theta)\right)
\end{align}
\section{DH frames}
\subsection{Assign axis}
\begin{itemize}
\item \textbf{$z_i$} along the direction of joint i+1.
\item \textbf{$x_i$} along the common normal between $z_i$ and $z_{i-1}$.
\item \textbf{$y_i$} completes the right-handed coordinate system.
\end{itemize}
\subsection{DH table}
\begin{itemize}
\item \textbf{$\theta_i$} angle between $x_{i-1}$ and $x_i$ measured about $z_{i-1}$.
\item \textbf{$d_i$} distance between $x_{i-1}$ and $x_i$ measured along $z_{i-1}$.
\item \textbf{$a_i$} distance between $z_{i-1}$ and $z_i$ measured along $x_i$.
\item \textbf{$\alpha_i$} angle between $z_{i-1}$ and $z_i$ measured about $x_i$.
\end{itemize}
\subsection{Transformation matrix from DH parameters}
\begin{equation}
    ^{i-1}A_i = \begin{bmatrix}
    \cos(\theta_i) & -\sin(\theta_i)\cos(\alpha_i) & \sin(\theta_i)\sin(\alpha_i) & a_i\cos(\theta_i) \\
    \sin(\theta_i) & \cos(\theta_i)\cos(\alpha_i) & -\cos(\theta_i)\sin(\alpha_i) & a_i\sin(\theta_i) \\
    0 & \sin(\alpha_i) & \cos(\alpha_i) & d_i \\
    0 & 0 & 0 & 1
    \end{bmatrix}
\end{equation}
\subsection{DH parameters from transformation matrix}
First we have to check that the first three by three submatrix is
 a rotation matrix
  \hyperref[sec:check_rotation]{(see section \ref{sec:check_rotation})}.\\
Then we can find the parameters:
\begin{align}
\theta_i &= \atan2\left(R_{12},R_{11}\right)\\
\alpha_i &= \atan2\left(R_{32},R_{33}\right)\\
d_i &= R_{34}\\
a_i &= R_{14}\cos(\theta_i)+R_{24}\sin(\theta_i)\\
\end{align}
 \section{Workspace}
\subsection{2-DOF robot}
The primary workspace is defined by two concentric circles of radius $r_1$ and $r_2$ where:
\begin{equation}
r_1 = |l_1 - l_2|
\end{equation}
\begin{equation}
r_2 = l_1 + l_2
\end{equation}
\subsection{3-DOF robot}
The primary workspace is defined by two concentric spheres of radius $r_{in}$ and $r_{out}$ where:
\begin{equation}
r_{out}= l_{min}+l_{med}+l_{max}
\end{equation}
\begin{equation}
r_{in} = \max(0,l_{max}- l_{med} -l_{min})
\end{equation}
where:
\begin{itemize}
\item $l_{min}$ is the length of the shortest link
\item $l_{med}$ is the length of the medium link
\item $l_{max}$ is the length of the longest link
\end{itemize}
\section{Inverse Kinematic}
\subsection{Trigonometry}
\begin{align}
    \cos(\theta+\phi) = \cos(\theta)\cos(\phi)-\sin(\theta)\sin(\phi) \\
    \sin(\theta+\phi) = \sin(\theta)\cos(\phi)+\cos(\theta)\sin(\phi)
\end{align}
\end{document}
